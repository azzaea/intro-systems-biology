\documentclass{article}
\newsavebox{\oldepsilon}
\savebox{\oldepsilon}{\ensuremath{\epsilon}}
\usepackage[minionint,mathlf,textlf]{MinionPro} % To gussy up a bit
\renewcommand*{\epsilon}{\usebox{\oldepsilon}}
\usepackage[margin=1in]{geometry}
\usepackage{graphicx} % For .eps inclusion
%\usepackage{indentfirst} % Controls indentation
\usepackage[compact]{titlesec} % For regulating spacing before section titles
\usepackage{adjustbox} % For vertically-aligned side-by-side minipages
\usepackage{array, amsmath,  mhchem}
\usepackage[hidelinks]{hyperref}
\usepackage{courier, subcaption}
\usepackage{multirow, enumerate}

\usepackage{float}
\restylefloat{table}

\pagenumbering{gobble} 
\setlength\parindent{0 cm}
\renewcommand{\arraystretch}{1.2}
\begin{document}
\large

MCB 135 Problem Set 4 \hfill Due Monday, March 2, 2015 at 2:30 PM

\section*{Problem 1: Positive Feedback (adapted from Alon 3.4, 15 points)}

In last week's problem set, you showed that a gene $Z$ with simple regulation decribed by
\begin{eqnarray*}
\frac{dZ}{dt} = \beta_0 - \alpha Z, \hspace{2 cm} Z(0)=0
\end{eqnarray*}
has a response time (i.e., time required to reach half of its steady-state expression level) given by $\tau=\frac{\ln 2}{\alpha}$. Now consider a protein $X$ which positively regulates its own expression:
\[ \frac{dX}{dt} = \beta_0 + \beta_1 X  - \alpha X, \hspace{1 cm} X(0)=0\]

\begin{enumerate}[a)]
\setlength{\itemsep}{0pt}
\item Identify any fixed points of this system and determine their stability. Consider the three cases (i) $\alpha < \beta_1$, (ii) $\alpha = \beta_1$, and (iii) $\alpha > \beta_1$.
\item When the response time of $X$ is well-defined, how does it compare to the response time of $Z$?
\item Describe a biological scenario where positive regulation of this type would be preferable to simple regulation. Explain your reasoning.
\end{enumerate}

%\section*{Problem 2: Transcription factor multimerization (adapted from Ingalls 7.8.3, 15 points)}
%
%Consider a gene $M$ which encodes monomers that dimerize to form a function transcription factor, $D$:
%\[ \ce{ M + M <=>[k_1][k_2] D} \]
%
%$D$ binds non-cooperatively to a single site in the promoter of $M$: $M$ is produced at a rate proportional to the probability that $D$ is bound. The dynamics of the system are described by:
%\begin{eqnarray*}
%\frac{d\left[ M \right]}{dt} & = & \alpha \frac{\left[ D \right]}{K + \left[ D \right]} - 2 k_1 \left[ M \right]^2 + 2 k_2 \left[ D \right] - \beta_M \left[ M \right]\\
%\frac{d\left[ D \right]}{dt} & = &  k_1 \left[ M \right]^2 - k_2 \left[ D \right] - \beta_D \left[ D \right]
%\end{eqnarray*}
%
%where $\alpha$ is the maximum production rate of [M] and $\beta_i$ are the degradation/dilution rates. Verify that if $ \beta_D=0$, then applying a quasi-steady state assumption to [D] yields:
%\[ \frac{d\left[ N \right]}{dt} = \alpha \frac{\left[ M \right]^2}{K' + \left[ M \right]^2} - \beta_M \left[ M \right] \]
%
%Thus, multimerization generates the same dynamics as cooperative DNA binding if multimers are protected from degradation.

\section*{Problem 2: Global dynamics from local stability analysis (Ingalls 4.8.6, 35 points)}

\begin{enumerate}[a)]
\setlength{\itemsep}{0pt}
\item Consider the chemical reaction network with mass-action kinetics:
\begin{eqnarray*}
\ce{A + X ->[k_1] 2 X} \hspace{2 cm} \ce{X + X ->[k_2] Y}  \hspace{2 cm} \ce{Y ->[k_3] B} 
\end{eqnarray*}
Assume that [A] and [B] are held constant.
\begin{enumerate}[i)]
\setlength{\itemsep}{0pt}
\item Write a differential equation model describing the concentrations of $X$ and $Y$.
\item Verify that the system has two steady states.
\item Determine the system Jacobian at the steady states and characterize the local behavior of the system near these points.
\item By referring to the network, provide an intuitive description of the system behavior from any initial condition for which [X] = 0.
\item Sketch a phase portrait for the system that is consistent with your expectations from (iii) and (iv).
\end{enumerate}
\item Repeat for the related system
\begin{eqnarray*}
\ce{A + X ->[k_1] 2 X} \hspace{2 cm} \ce{X + Y ->[k_2] 2Y}  \hspace{2 cm} \ce{Y ->[k_3] B} 
\end{eqnarray*}
In this case, you'll find that the nonzero steady state is a center: it is surrounded by concentric periodic trajectories.
\end{enumerate}

\section*{Problem 3: Practice with Laplace transforms (adapted from Meister 2009, 50 points)}

Absorption of a single photon by a photoreceptor neuron triggers a biochemical cascade that results in a change in current across the neuron's membrane called the ``single photon response," which lasts for a few hundred milliseconds before decaying away. The neuron's photon detection system is approximately linear and time-invariant: when several photons get absorbed, the resulting current is simply the sum of all their single photon responses.

\begin{enumerate}[a)]
\setlength{\itemsep}{0pt}
\item  We will represent the single photon response for a photon absorbed at time $t=0$ by $h(t)$. What is the photoreceptor's response $O(t)$ to an arbitrary time-varying light stimulus, in which the rate of photon absorptions varies with time as $I(t)$, where
\[ I(t) \, dt = \textrm{number of photons absorbed in short interval } dt \textrm{ ?} \]
Write the answer as a convolution integral, then apply a Laplace transform to find $\tilde{O}(s)$ in terms of $\tilde{I}(s)$ and $\tilde{h}(s)$.
\item One way to investigate a system's behavior is to apply a well-defined ``test" input and study the resulting output timecourse. Determine the Laplace transforms $\tilde{I}(s)$ for the two common types of test inputs given below, showing your work:
\begin{enumerate}[i)]
\setlength{\itemsep}{0pt}
\item  $I(t) = \delta(t)= \left\{
     \begin{array}{lr}
       \infty, & t = 0 \\
       0, & t \neq 0
     \end{array}
   \right.$ (Dirac delta/unit impulse)
\item  $I(t) = \theta(t)= \left\{
     \begin{array}{lr}
       0, & t < 0 \\
       1, & t \geq 0
     \end{array}
   \right.$ (Heaviside step function)
\end{enumerate}
\end{enumerate}
When a very brief flash of light consisting of $N$ photons is delivered to a photoreceptor, the photon detection system's output is:
\[ O(t) = \left\{
     \begin{array}{lr}
       0, & t < 0 \\
       A \sin \left( \omega t \right) e^{-\alpha t}, & t \geq 0
     \end{array}
   \right. \]   
   \begin{enumerate}[a)]
\setlength{\itemsep}{0pt}
\setcounter{enumi}{2}
\item Write an expression for $I(t)$ in this case, and then apply a Laplace transform to find an expression for $\tilde{I}(s)$.
\item Find an expression for $\tilde{O}(s)$ (for the rest of the problem, you may consult a table of Laplace transforms\footnote{A thorough table is available at \url{http://www.dartmouth.edu/\~sullivan/22files/New\%20Laplace\%20Transform\%20Table.pdf}.}). 
\item Find the impulse response $h(t)$ by determining $\tilde{h}(s)$ and finding the inverse Laplace transform.
\end{enumerate}
An input of the form
\[ I(t) = \left\{
     \begin{array}{lr}
       0, & t < 0 \\
      M \textrm{ photons per second, } & t \geq 0
     \end{array}
   \right. \]      
is then applied. Below you will determine what output $O(t)$ was expected under the assumption that the photon detection system is linear and time-invariant.
\begin{enumerate}[a)]  
\setlength{\itemsep}{0pt}
\setcounter{enumi}{4}
\item What is $\tilde{I}(s)$?
\item Use $\tilde{h}(s)$, calculated in part (e), to express $\tilde{O}(s)$ as a rational function in $s$.
\item Determine $O(t)$. What value does $O(t)$ approach as $t \to \infty$?
\end{enumerate}

\end{document}